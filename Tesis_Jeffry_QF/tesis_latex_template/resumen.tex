\chapter*{Resumen}
Este trabajo es la continuación del desarrollo de una unidad aritmética de coma flotante(FPU), iniciada el semestre anterior, basada en el estándar \textit{IEEE 754}. Para esta etapa se desarrollaron los módulos de cálculo de las operaciones trigonométricas seno y coseno utilizando arquitecturas de 32 y 64 bits. Las operaciones trigonométricas se implementaron en una placa de prototipos (FPGA) utilizando el lenguaje de descripció de hardware(HDL) Verilog. Se realizó una verificación funcional de las operaciones a nivel de simulación y sobre la FPGA, lograndose un porcentaje de error menor al $ 1\% $, respecto al modelo teórico de referencia elaborado en el software \textit{GNU Octave}. Finalmente, se presentan los resultados obtenidos de las simulaciones \textit{Post-Implementation}, referentes al porcentaje de error en los cálculos, consumo de potencia, reportes de tiempos y uso de recursos de hardware.



\bigskip

\textbf{Palabras clave:} \scriptKeywords

\clearpage
\chapter*{Abstract}
\thispagestyle{empty}



\bigskip

\textbf{Keywords:} \scriptKeywordsEnglish %CORDIC, sine, cosine, FPU, simple precision, double precision, FPGA. 

\cleardoublepage

%%% Local Variables: 
%%% mode: latex
%%% TeX-master: "main"
%%% End: 
